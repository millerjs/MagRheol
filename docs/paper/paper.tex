
%% ----------- [ Shade table
% \shadetable{c|c}{column1 & column2}
% {Test Table 1}{  
%   a & b \\
%   c & d
% }{tab:test1}


%% ----------- [ Figs
% \figH {FILENAME}{CAPTION}{LABEL}{WIDTH}       


%% ----------- [ Matrices
% \m{row1}[{row2}][{...}{...}...]



\documentclass[12pt]{article}
\usepackage[table]{xcolor}
\usepackage{fancyhdr}
\usepackage{amsmath} 
\usepackage[margin=3cm]{geometry}
\usepackage{graphicx} 
\usepackage{longtable}  
\usepackage{multicol} 
\usepackage{float}
\usepackage{xparse}
\usepackage{amssymb}
\newcommand{\degrees}{\ensuremath{^\circ}}
\usepackage{wrapfig}
\usepackage{parskip}
\usepackage{listings}
\usepackage{enumerate}
\usepackage{mathtools}
\usepackage{hyperref}

%% \usepackage{courier}


\usepackage{inconsolata}
 %% \usepackage[charter]{mathdesign}
 %% \def\rmdefault{bch} % not scaled
 %% \def\ttdefault{blg}

\newcommand{\csection}[1]{\section[#1]{\centering #1}}
\newcommand{\csubsection}[1]{\subsection[#1]{\centering #1}}
\newcommand{\bv}{\begin{verbatim}}
\newcommand{\ev}{\end{verbatim}}
\renewcommand{\=}{&=}
\renewcommand{\k}{{\k_B}}

\renewcommand{\d}{\ensuremath{\text{d}}}
\newcommand{\cin}[1]{{\scriptsize \lstinputlisting{#1}}}

\newcommand{\lang}[1]{\lstset{language=#1}}
\newcommand{\import}[2]{\lstset{language=#1}{\scriptsize \lstinputlisting{#2}}}
\lstnewenvironment{code} {} {}


\DeclarePairedDelimiter{\ceil}{\lceil}{\rceil}
\DeclarePairedDelimiter{\floor}{\lfloor}{\rfloor}


\usepackage{fancyvrb}

\RecustomVerbatimEnvironment
{Verbatim}{Verbatim}
{frame=single}

\renewcommand{\exp}[1]{\ensuremath{e^{#1}}}

\newcommand{\del}{\vec\nabla}
\newcommand{\p}{\partial}
\newcommand{\E}[1]{\times10^{#1}}
\newcommand{\eq}[1]{\begin{equation*}#1\end{equation*}}
\newcommand{\al}[1]{\begin{align*}#1\end{align*}}
\newcommand{\aln}[1]{\begin{align}#1\end{align}}
\newcommand{\sub}[1]{_\text{#1}}
\newcommand{\vv}{\mathbf}


\newcommand{\dd}[2]{\frac{\d\,#1}{\d\,#2}}
\newcommand{\ddo}[1]{\frac{d}{d\,#1}}
\newcommand{\ddt}[1]{\frac{\d\,#1}{\d\,t}}
\newcommand{\dds}[2]{\frac{\d^2\,#1}{\d\,#2^2}}
\newcommand{\pd}[2]{\frac{\partial \,#1}{\partial \,#2}}
\newcommand{\spd}[2]{\frac{\partial^2 #1}{\partial #2^2}}
\newcommand{\lp}{\left(}
\newcommand{\rp}{\right)}

\newcommand{\lb}{\left[}
\newcommand{\rb}{\right]}


\usepackage{color}

\definecolor{dkgreen}{rgb}{0,0.4,0}
\definecolor{redd}{rgb}{.6,0.1,0.1}
\definecolor{bluee}{rgb}{.1,.1,.8}
\definecolor{white}{rgb}{1.,1.,1.}
\definecolor{gray}{rgb}{0.5,0.5,0.5}
\definecolor{grayy}{rgb}{0.95,0.95,0.95}
\definecolor{mauve}{rgb}{0.58,0,0.82}


\lstset{
  %% language=C,          
  keywordstyle=\color{bluee}\sffamily\bfseries,  
  commentstyle=\color{dkgreen}\itshape,    
  backgroundcolor=\color{Yellow!25},
  basicstyle=\footnotesize\ttfamily,
  breaklines=true,
  %% numbers=left,              
  %% numberstyle=\footnotesize,      
  %% stepnumber=5,                   
  %% numbersep=15pt,                  
  backgroundcolor=\color{grayy},  
  showspaces=false,               
  showstringspaces=false,         
  showtabs=false,                 
  frame=single,           
  %% framerule=15pt,%expands outward 
  rulecolor=\color{white},
  % framesep=3pt,%expands outward
  tabsize=4,          
  captionpos=b,           
  breaklines=true,        
  breakatwhitespace=true,    
  xleftmargin=20pt,
  xrightmargin=20pt,
}

\definecolor{lightblue}{rgb}{0.83,0.85,1.0}

\newcommand{\shadetable}[5]
{
  \begin{table}[H] 
    \begin{center}
      \caption{#3} \vspace{5pt}
      \rowcolors{1}{}{lightblue}
      \begin{tabular}{#1} \hline 
        #2 \\
        \hline
        #4 \\
        \hline                       
      \end{tabular}
      \label{#5}
    \end{center}
  \end{table}
}
% \shadetable{c|c}{column1 & column2}
% {Test Table 1}{  
%   a & b \\
%   c & d
% }{tab:test1}

\DeclareDocumentCommand\mb{ m g g g g g g g g}
  {
      \begin{smallmatrix} 
        #1
        \IfNoValueF{#2}{\\#2}
        \IfNoValueF{#3}{\\#3}
        \IfNoValueF{#4}{\\#4}
        \IfNoValueF{#5}{\\#5}
        \IfNoValueF{#6}{\\#6}
        \IfNoValueF{#7}{\\#7}
        \IfNoValueF{#8}{\\#8}
        \IfNoValueF{#9}{\\#9}
      \end{smallmatrix} 
}
% \m{row1}[{row2}][{...}{...}...]



\DeclareDocumentCommand\m{ m g g g g g g g g}
  {
    \left( 
      \begin{smallmatrix} 
        #1
        \IfNoValueF{#2}{\\#2}
        \IfNoValueF{#3}{\\#3}
        \IfNoValueF{#4}{\\#4}
        \IfNoValueF{#5}{\\#5}
        \IfNoValueF{#6}{\\#6}
        \IfNoValueF{#7}{\\#7}
        \IfNoValueF{#8}{\\#8}
        \IfNoValueF{#9}{\\#9}
      \end{smallmatrix} 
    \right)
}
% \m{row1}[{row2}][{...}{...}...]

\newcommand{\fig}[4]
{
  \begin{figure}[h] \centering \vspace{-10pt} 
    \includegraphics[width=#4\textwidth]{#1} \vspace{-10pt} 
    \caption{#2} \label{#3}
  \end{figure}
}
% \figH {FILENAME}{CAPTION}{LABEL}{WIDTH}       



\newcommand{\figH}[4]
{
  \begin{figure}[H] \centering \vspace{-10pt} 
    \includegraphics[width=#4\textwidth]{#1}\vspace{-10pt} 
    \caption{#2} \label{#3}
  \end{figure}
} 
% \figH {FILENAME}{CAPTION}{LABEL}{WIDTH}       


% \AtBeginDocument{%
  % \setlength\abovedisplayskip{0pt}
  % \setlength\belowdisplayskip{0pt}}

% ------------------------------------------------------------------------
%                               Page Style
% ------------------------------------------------------------------------

\pagestyle{fancy}
\renewcommand{\sectionmark}[1]{\markright{\thesection\ #1}}
\fancyhf{}

\lhead{\fancyplain{}{\rightmark }}
\cfoot{\fancyplain{}{\thepage}}


%% ------------------------------------------------------------------------
%%                                  MIPS
%% ------------------------------------------------------------------------

\lstdefinelanguage[mips]{Assembler}{%
  % so listings can detect directives and register names
  alsoletter={.\$},
  % strings, characters, and comments
  morestring=[b]'',
  morestring=[b]',
  morecomment=[l]\#,
  % instructions
  morekeywords={[1]abs,abs.d,abs.s,add,add.d,add.s,addi,addiu,addu,%
    and,andi,b,bc1f,bc1t,beq,beqz,bge,bgeu,bgez,bgezal,bgt,bgtu,%
    bgtz,ble,bleu,blez,blt,bltu,bltz,bltzal,bne,bnez,break,c.eq.d,%
    c.eq.s,c.le.d,c.le.s,c.lt.d,c.lt.s,ceil.w.d,ceil.w.s,clo,clz,%
    cvt.d.s,cvt.d.w,cvt.s.d,cvt.s.w,cvt.w.d,cvt.w.s,div,div.d,div.s,%
    divu,eret,floor.w.d,floor.w.s,j,jal,jalr,jr,l.d,l.s,la,lb,lbu,%
    ld,ldc1,lh,lhu,li,ll,lui,lw,lwc1,lwl,lwr,madd,maddu,mfc0,mfc1,%
    mfc1.d,mfhi,mflo,mov.d,mov.s,move,movf,movf.d,movf.s,movn,movn.d,%
    movn.s,movt,movt.d,movt.s,movz,movz.d,movz.s,msub,msubu,mtc0,mtc1,%
    mtc1.d,mthi,mtlo,mul,mul.d,mul.s,mulo,mulou,mult,multu,mulu,neg,%
    neg.d,neg.s,negu,nop,nor,not,or,ori,rem,remu,rol,ror,round.w.d,%
    round.w.s,s.d,s.s,sb,sc,sd,sdc1,seq,sge,sgeu,sgt,sgtu,sh,sle,%
    sleu,sll,sllv,slt,slti,sltiu,sltu,sne,sqrt.d,sqrt.s,sra,srav,srl,%
    srlv,sub,sub.d,sub.s,subi,subiu,subu,sw,swc1,swl,swr,syscall,teq,%
    teqi,tge,tgei,tgeiu,tgeu,tlt,tlti,tltiu,tltu,tne,tnei,trunc.w.d,%
    trunc.w.s,ulh,ulhu,ulw,ush,usw,xor,xori},
  % assembler directives
  morekeywords={[2].align,.ascii,.asciiz,.byte,.data,.double,.extern,%
    .float,.globl,.half,.kdata,.ktext,.set,.space,.text,.word},
  % register names
  morekeywords={[3]\$0,\$1,\$2,\$3,\$4,\$5,\$6,\$7,\$8,\$9,\$10,\$11,%
    \$12,\$13,\$14,\$15,\$16,\$17,\$18,\$19,\$20,\$21,\$22,\$23,\$24,%
    \$25,\$26,\$27,\$28,\$29,\$30,\$31,%
    \$zero,\$at,\$v0,\$v1,\$a0,\$a1,\$a2,\$a3,\$t0,\$t1,\$t2,\$t3,\$t4,
    \$t5,\$t6,\$t7,\$s0,\$s1,\$s2,\$s3,\$s4,\$s5,\$s6,\$s7,\$t8,\$t9,%
    \$k0,\$k1,\$gp,\$sp,\$fp,\$ra},
}[strings,comments,keywords]


\title{Modeling Magnetorheological Fluids \\ {\small CHEM 26800 Computational Chemistry and Biology \\Sprint 2014}} 
\author{Joshua Miller}
\date{ }

\begin{document}
\rhead{\fancyplain{}{Joshua Miller}}
\maketitle

\abstract{ 

The purpose of this project was to create a simple model of a
magnetorheological fluid in order to examine the structural trends as
a function of magnetic to non-magnetic particle ratio and external
field strength.  The rheological properties were examined by testing
the ability of a massive projectile to penetrate the fluid.

}

%% \begin{multicols*}{2}

\section*{Introduction}

Magnetorheological fluids, similar to their ferrofluid cousins, have
the ability to be controlled by magnetic fields. Composed of a
micrometer sized magnetic particles suspended in a non-magnetic
medium, the application of an external magnetic field induces two
conformational changes. A magnetic particle experiences an interaction
force with those surrounding it, but the random alignment of the
magnetic poles prevents this force from creating a global structure
amongst them. With the application of an external magnetic field, the
torsion aligns the magnetic along the field.  This uniformity causes
the magnetic pole interactions to create chain-like structures amongst
the magnetic particles. These chain structures increase the apparent
magnitude of the fluid and may cause the fluid to become a
viscoelastic solid.  This property gives MR fluids applications in
vehicle shock absorbers, prosthetics, and impact resistant
materials. The goal of creating a model of MR fluids was to look at
trends in its rheological properties by examining the viscosity under
varying magnetic field strengths as well as varying magnetic to
non-magnetic particle ratios.

The simulation was written in C and is available at
\url{github.com/millerjs/MagRheol}. There is a preprocessor definition
that relies on {\tt GNU\_SOURCE} and therefore the code has only been
tested on a GNU linux machine.

\section*{Fluid Model}

The model presented on this paper is based on the work of Y. Ido,
T. Inagaki, and T. Yamaguchi\cite{1}.  For simplicity, the model fluid
is composed of uniform, hard, spherical particles suspended in a
carrier fluid.  The magnetic particles are approximated to be dipoles.
Particle parameters used in this simulation mirror those in the Ido
simulation. The magnetic material is approximated to be ZnFe$_2$O$_4$
in particles of radius $1.5 \mu$m with a density of $\rho = 2.5\times10^{-4}$
kg~m$^{-3}$. The non-magnetic particles maintain the same properties
aside from the lack of magnetic dipole moment.  

\subsection*{Domain and Magnetic Field}

This simulation ignores the effect of the fluid on the magnetic field,
and the field strength and direction is considered constant throughout
the domain. The domain was constructed by selecting the number of
particles used in the simulation, a cubic number for convenience, and
using this to determine the domain size. For example, for a simulation
with 343 particles, the domain volume was approximated as

\al{ 
\text{V} &= \frac{\text{no. of particles}}{\text{particle dens}} \\
\text{particle dens} &= \frac{\rho}{m_\text{particle}} \\
m_\text{particle} &= m_\text{molecule}\times\frac{\text{molecules}}{\text{particle}}\\
m_\text{molecule} &= \frac{m_{ZnFe_2O_4}}{N_A}\\
\frac{\text{molecules}}{\text{particle}} &\approx \frac{r_{particle}^3}{r_{molecule}^3} 
}

This calculation provided for a cubic domain on the order of $40 \mu$m
in length.  The external magnetic field was $- H\hat z$ across the
domain. During the initial creation of the model the boundary
condition {\tt PERIODIC} was applied to all faces of the domain.  All
distances calculated for directions with periodic faces used the
minimum image convention. I found that the periodic boundary condition
was especially important for the faces perpendicular to magnetic
field.  This not only allowed the fluid to behave as if it were in a
larger domain but also prevented the truncation of particle chains
near the boundaries that would be seen in a non-physically small
domain from a lack of dipole attraction across the border. This
condition is necessary because the dipole interaction force was not
mirrored across a reflecting boundary condition in this
simulation. The {\tt REFLECTING} boundary condition was created simply
by reflecting the velocity of particles crossing the face. 

The code reads in a parameter config file to avoid compilation for every new
initial setup. Included below is a sample config file:
\begin{verbatim}
# magrheol config file
# maxt in 1e-4 seconds
maxt=                 12.
checkpoint_interval=  50
dt=                   .001
# Particle counts/ratio
npart=                343
ratio=                .5
# dimensions in micrometers
X=                    4.223095e+01
Y=                    4.223095e+01
Z=                    4.223095e+01
# Magnetic field and dipole
# mTesla
H=                    70
MU=                   .01
\end{verbatim}


\subsection*{Forces/Torques}

Due to the reduction of the magnetic particles to spheres with dipole
moments, the forces needed to simulate the fluid included
$F_\text{rep}$, the physical repulsion force,
$F_\text{Dipole-Dipole}$, the dipole interaction between magnetic
particles, and $F_\text{drag}$. Because the particles are on the
micrometer scale, the random force due to Brownian motion was
considered negligible. The forces used were as derived in the Ido
paper\cite{1}. The repulsion force was based on the DLVO theory which
is derived from the van der Waal's attraction and the Double Layer
force caused by charge dissociation at particle surfaces. 

\al{
 F_{rep} = \frac{3\mu_s^2}{64\pi\mu_0R^4} e^{-\kappa S}
  \frac{\vec r}{||\vec r||}
}
where $S$ is the distance between the surfaces of the two particles
\footnote{This repulsion convention was novel to me.  I found it
  extremely interesting because of its generality over surface
  topologies.} and is the inverse Debeye screening length, with a
value provided in the Ido paper\cite{1}. The Dipole-Dipole interaction force
was
\al{
 F_{Dipole} &= \frac{3}{4\pi\mu_0||\vec r||^4}  \bigg[ (\mu_i\cdot \mu_j) \frac {\vec r}{||\vec r||} - 5(\mu_i\cdot\vec r )(\mu_j\cdot \vec r)\frac {\vec r}{||\vec r||}  + \lp (\mu_i\cdot \vec r)\mu_j + (\mu_j\cdot \vec r\mu_i) \rp \frac1{||\vec r||} \bigg ].
}
The drag force was applied with Stokes' drag,
\al{
 F_{drag} = -6\pi\eta r \vec v,
}
where the particle radius was used for the Stokes radius. In order for
the particles to align to the magnetic field, the must experience a
torque.  The torque on a magnetic dipole in an external field is
\al{
T_H &= \vec \mu \times \vec H.
}
Furthermore each magnetic particle experiences a torsion force from
the dipoles around it,
\al{
T_{Dipole} = \frac{1}{4\pi\mu_0||\vec r||^3} \left ( \mu_i\times \mu_j - \frac 3 {||\vec r||^2} (\mu_j\cdot \vec r) \mu_i \times \vec r\right ).
}


\subsection*{Evolving the state}

%% Initial

Both the positions and the orientation of the dipole moment were
evolved using Verlet integration. Because long evolution and the small
time step that were necessary, I decided to parallelize the verlet
algorithm in order to utilize any available cores. I created a pool of
{\tt pthreads}.  Each thread performs $\frac1N$th of the forces and
verlet integration where $N$ is the number of threads set to be
used. The inclusion of this scheme necessitated an update in the
integration. Due to the race condition introduced by the threading,
the data belonging to thread was not guaranteed to be at the right
time step when viewed by another thread.  To handle this, the data was
stored in pre-allocated temporary arrays.  Each thread utilized a
thread barrier to wait until all threads had updated their positions
to a temporary memory location before the master thread performed
calculations involving projectile position and performed output writes
to checkpoints.  After the thread barrier was reached, the variables
pointing to the (old and new) position and dipole vectors were
rotated. Using 16 cores I was able to achieve approximately a 1200\%
speedup.  Even with this parallelization, simulations were relatively
costly, making it difficult to verify parameters such as the time step
and gather quantitative data on the rheological properties of the
fluid.

In order to test the rheological properties of the fluid, a massive
particle was projected into the domain, and propelled by a small
force.  This process is akin to measuring the viscosity of a fluid by
calculating the rate of descent of an object under a gravitational
field. The strength of the external force on the projectile was
determined empirically by testing which forces provided stable motion
at a reasonable time step while still eventually overcoming any
Brownian opposition force on the projectile by the simulation
particles. The time required for the particle to travel through the
domain was used as a metric of the rheological properties given a
magnetic field strength and particle ratio.

The simulation code runs by taking a configuration file as an input
and outputs three different data files at each checkpoint interval
containing information about (1) all the particles (2) the magnetic
particles and (3) the projectile.

\subsection*{Results}

Figures~\ref{initial}~\&~\ref{final} demonstrate the initial state
$t=0$ms and the final state $t=2.0$ms for a simulation with 343
particles.

\begin{multicols}{2}
\figH{initial.png}{Initial Setup with magnetic (blue) and non-magnetic
  (white).}{initial}{.4}

\figH{final.png}{Final state after $t=2.0$ms. Magnetic (blue) and non-magnetic
  (white).}{final}{.4}
  \end{multicols}

In order to quantitatively measure the convergence pattern of the
chaining effects the k-means clustering algorithm\cite{4} was used to
ascribe a chain to the particle locations projected onto X-Y plane (as
it is perpendicular to the magnetic field). For a simulation with a
cubic domain of side length 42.2$\mu$m, there can be 14 molecules
chained along the magnetic field direction in the domain.  Therefore,
for 343 particles, we would expect to see approximately 13
clusters. Figure~\ref{kmeans} below demonstrates the distortion (the
displacement of each point to the centroid it has been classified to)
of the points after applicaiton of the k-means algorithm as a function
of time.

\figH{../../analysis/plt2.png}{K-Means clustering distortion over time, 
$K=13$, $dt = 1\E{-6} $ ms} {kmeans}{.8}

An exponential function was applied using a linear error regression of
the form $\text{Distortion} = A e^{-Bt} + C$.  For this analysis we
are interested in the decay rate $B$. For the example above, the decay
rate was $B = 0.00567192~s^{-2}$.


\newpage
\vspace{40pt}
\begin{thebibliography}{9}


\bibitem{1}
Y. Ido, T. Inagaki, and T. Yamaguchi.
\emph{Numerical simulation of microstructure formation of suspended particles in magnetorheological fluids.}  
Journal of Physics: Condensed Matter. 
IOP Publishing.
15 July 2010. 

\bibitem{2}
Spinks, Joseph Michael. 
\emph{Dynamic simulation of particles in a magnetorheological fluid.}
PhD diss., Monterey California. 
Naval Postgraduate School, 2008.

\bibitem{3}
ViSit Visualization, animation and analysis tool.
\newline 
\url{https://wci.llnl.gov/codes/visit/}

\bibitem{4}
SciPy k-means. 
\url{http://docs.scipy.org/doc/scipy-0.13.0/reference/generated/scipy.cluster.vq.kmeans.html}.
The Scipy community. Copyright 2008-2009.


\end{thebibliography}



%% \end{multicols*}


\end{document}
